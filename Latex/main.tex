\documentclass{article}
\usepackage{ multicol } 
\usepackage[letterpaper,top=1cm,bottom=2cm,left=3cm,right=3cm,marginparwidth=1.75cm]{geometry}
\usepackage{amsmath}
\usepackage{graphicx}
\usepackage[utf8]{inputenc}
\usepackage[english]{babel}
\usepackage[colorlinks=true, allcolors=blue]{hyperref}
\usepackage{amssymb}
\providecommand{\abs}[1]{\lvert#1\rvert}
\providecommand{\norm}[1]{\lVert#1\rVert}
\DeclareUnicodeCharacter{2212}{-}

\title{\textbf{Dwarf spheroidal galaxies and Bose-Einstein condensate dark matter}}
\author[1]{Alberto Diez-Tejedor}
\author[2]{Alma X. Gonzalez-Morales}
\author[3]{Stefano Profumo}
\affil {Santa Cruz Institute for Particle Physics and Department of Physics,University of California, Santa Cruz, CA, 95064, USA}

\begin{document}
\maketitle
\begin{abstract}
We constrain the parameters of a self-interacting massive dark matter scalar particle in a con-densate  using  the  kinematics  of  the  eight  brightest  dwarf  spheroidal  satellites  of  the  Milky  Way.For  the case  of a  repulsive  self-interaction  the condensate  develops  a mass  density  profile with  acharacteristic scale radius that is closely related to the fundamental parameters of the theory.  Wefind that the velocity dispersion of dwarf spheroidal galaxies suggests a scale radius of the orderof 1 kpc, in tension with previous results found using the rotational curve of low-surface-brightnessand dwarf galaxies.  The new value is however favored marginally by the constraints coming fromthe number of relativistic species at Big-Bang nucleosynthesis.  We discuss the implications of ourfindings for the particle dark matter model and argue that while a single classical coherent state cancorrectly describe the dark matter in dwarf spheroidal galaxies, it cannot play, in general, a relevantrole for the description of dark matter in bigger objects.
\end{abstract}
\begin{multicols} {2}
\section{INTRODUCTION}

The nature of dark matter (DM) remains an open question.   At  the  fundamental  level,  DM  is  expected  to  be described  in  terms  of  a  quantum  field  theory.   At  the effective  level,  however,  a  description  in  terms  of  classical  particles  is  usually  considered,  see  e.g.   the  large literature  on  N-body  simulations  \cite{Boylan_Kolchin_2009}[1].   Most  current  efforts are focused on detecting a weakly interacting massive particle (WIMP), both by direct \cite{Kim_2013}[2] and indirect \cite{Profumo_2012} [3]searches.   In  the  case  of  WIMPs  its  present-day  abun-dance is fixed at the time when DM decoupled from the thermal plasma.  If the interaction of DM lies at the weakscale, with a mass of the particle in the range of 100 GeV(as expected from the super symmetric extensions to the standard model), the energy density of these particles co-incides “miraculously” with the observed one \cite{Bertone_2005}[4].  How-ever,  alternatives  exist  and  deserve  careful  scrutiny,  either  to  constrain  the  associated  parameter  space,  andthus phenomenology, or to dismiss them as viable candidates.One such proposal considers that the abundance of DMis fixed by an asymmetry between the number densitiesof particles and antiparticles \cite{PETRAKI_2013}[5], similarly to the baryonsand leptons in the universe.  If the particle interactions inthe early universe are strong enough to guarantee ther-mal equilibrium, and DM is further composed of a spin-0quantum  field,  the  zero  mode  could  have  developed  aBose-Einstein  condensate  where  a  description  in  terms of a classical field would be warranted.  Classical coherent states can also emerge non thermally, no asymmetry required, by means of the vacuum misalignment mechanism [6]. Similar ideas have been considered previously inthe literature under many different names, such as scalar field  [7],  BEC  [8],  Q-ball  [9],  fuzzy  [10],  boson  [11],  oreven fluid [12], DM; see also Refs. [13–23] for details.A natural realization of this scenario can be providedby  the  axion  [14].   Originally  introduced  to  solve  thecharge-parity violation problem in QCD [24], the axion was  soon  recognized  as  a  promising  candidate  for  DM.In  this  case  the  size  of  the  condensate  is  so  small  [25]that, most probably, DM halos made of axion-balls could not  be  distinguished  from  the  ones  simulated  with  N-body codes by means of galactic dynamics and/or lens-ing  observations  [26].   Another  possibility  is  that  withan  appropriate  choice  of  the  parameters  in  the  model(see  the  next  two  paragraphs  for  details),  it  could  be possible  to  develop  single  structures  with  the  size  of  a galaxy [8, 11, 15–20, 22, 23].For  practical  purposes  we  will  restrict  our  attention to the case of a massive, self-interacting, complex scalarfield  with  an  internalU(1)  global  symmetry  satisfying the Klein-Gordon equation2 
\begin{equation}
{2\phi−{(\frac{mc}{h})}{2\phi-2\lambda{\abs{\phi}{2\lambda}}}}
\end{equation}
Here the box denotes the d’Alembertian operator in four dimensions, withmthe mass of the scalar particle and $\lambda$ a dimensionless self-interaction term.  As long as the interaction between bosons is repulsive,$\lambda>0$, a universal  mass  density  profile  for  the  static,  spherically  symmetric, regular, asymptotically flat, self-gravitating equilibrium  scalar  field  configurations  emerges  in  the  weak field, Thomas-Fermi regime [8, 16–18, 27] of the Einstein-Klein-Gordon system with the following analytic form:
\begin{equation}
 {\rho(r)} = \left\{ \begin{array}{lcc}
             {\rho c}{\frac{\sin(\pi r/r_{max})}{(\pi r/r_{max})}} for &r < r_{max}\\
             \\ 0 for &r{\geq}r_{max}.
             \end{array}
   \right.
\end{equation}
In the effective description above there are two free pa-rameters:  first, the size of the gravitating objects,
\begin{equation}
 r_{max}\equiv{\sqrt{\frac{r{\pi}^{2}{\Lambda}}{2}}{\left(\frac{h}{mc}\right)}= 48.93{\left(\frac{\lambda^{1/4}}{m[eV/c^2]}\right)}}^2{kpc},  
\end{equation}
a parameter that, as manifest from the equation, depends directly on the bare constants of the theory in the combination $(m/\lambda)^{1/4}$;  second, the value of the mass density at the center of the configuration,$\rho c\equiv \pi mQ/(4r^3_{max})$, a quantity that in principle can vary from galaxy to galaxy.Here $Q$ is  the  total  charge  in  the  system,  that  in  this case coincides with the total number of particles, and for convenience we have defined the dimensionless constant $\Lambda \equiv \lambda m^/4\pi m^2$.The mass density profile in Eq. (2) describes only the diluted configurations of a scalar field in a regime of weak gravity; in terms of particle numbers that translates into(see e.g.  Eq. (25) in Ref. [15])
\begin{equation}
{\Lambda^{-1/2}}\ll {\left(\frac{m}{m_Plank}\right)}^2{Q}\ll {\Lambda^{1/2}}.
\end{equation}
The inequalities in Eq. (4) demand $\Lambda \gg1$; that is guaranteed if the combination $m/\lambda^{1/4}$ for the mass and self-interaction  terms  of  the  scalar  boson  is  well  below  the Planck  scale.    It  is  precisely  the  very  large  value  expected for the constant $\Lambda$ what makes possible to blowup the Compton wavelength of the scalar particle, $h/mc$, up  to  galactic  scales,  see  Eq.  (3)  above. Only  configurations  with  masses $M=mQ$ in  the  range  from $M\gg \lambda^{−1/2}m_{Planck}$ up to $M\ll \lambda^{1/2}m^{3}_{Planck}/m^2$ can be described by the expression in Eq. (2).  Then, in order to have an halo model for objects of at least $M\sim10^{8}M(M\sim10^{12}M)$ we  need  a  scalar  DM  particle  with $m/\lambda^{1/4}<70keV/c^2(m/\lambda^{1/4}<0.7keV/c^2)$.The  density  profile  in  Eq.  (2)  was  derived  under  the assumption  that  all  the  DM  particles  are  in  a  condensate,  while  in  a  more  realistic  situation  probably  only a  fraction  of  them  would  be  represented  by  the  coherent classical state.  (That seems indeed necessary in order to explain the flattened rotation curves in large spirals, where observations suggest $\rho \sim1/r^2$ at large radii.)Unfortunately, there is not yet a satisfactory description that includes this effect (see Ref. [28] for a proposal in this  direction).   Nevertheless,  this  halo  model  can  still be deemed appropriate to test the self-interacting scalar field DM scenario if we carefully choose observations that are sensitive only to the mass contained up to a radius smaller or comparable to $r_{max}$, where the condensate is expected to dominate the distribution of DM. One should then look at the profile in Eq. (2) not necessarily as a DM halo model for the whole galaxy, but for the core of the self-gravitating object only.The  dwarf  spheroidal  (dSph)  satellites  of  the  Milky Way  are  probably  the  most  promising  objects  to  test DM  models  as  far  as  structure  formation  is  concerned.These  old,  pressure-supported  systems  are  the  smallest and least luminous known galaxies,  and there is strong evidence that they are DM-dominated at all radii, with mass-to-light ratios as large as [29]
\begin{equation}
M/L_V \sim 10^{1−2}[M/LV]
\end{equation}
The dynamics of these objects, for instance, could allow us to determine whether DM halos are cored or cuspy:since  the  concentration  of  baryons  in  these  galaxies  is so low, effects such as the adiabatic contraction and/or supernova feedback cannot alter significantly the shape of the original halo.  Current data do not yet conclusively discriminate between cuspy and cored profiles [30–33].In this paper we use the kinematics of the eight classical  dSph  satellites  of  the  Milky  Way  to  determine whether a self-interacting scalar particle in a condensate is able to reproduce the galaxies’ internal dynamics and,if so, under what conditions on the theory input parameters.  In this respect, our study extends previous analyses carried out for the generalized Hernquist [30] and Burk-ert [31] profiles to the DM halo model in Eq. (2).  It is important to note, however, that the purpose of this pa-per is not to compare the profile in Eq. (2) with other halo  models  in  the  literature,  but,  rather,  to  use  dSph dynamics  to  test  the  self-consistency  of  the  scalar  field dark matter scenario.We find that the eight classical dSphs indicate a scale radius of the order 
\begin{equation}
r_{max}\sim 1 kpc,~i.e.~{m/\lambda^{1/4}\sim {7eV/c^2}},
\end{equation}
a  value  in  tension  with  previous  results  found  using the rotation curves of low-surface-brightness (LSB) and dwarf galaxies [8, 16, 18, 19];  see also Refs. [22, 23] for bigger  galaxies.   Our  findings  strongly  disfavor  a self-interacting condensate DM halo model or, if one hypo the-sizes that the condensate describes only the core of galaxies, they indicate that the relevance of the coherent stateto describe DM in larger galaxies is, at best, negligible.

\section{THE JEANS EQUATION}
Dwarf spheroidal galaxies are simple, old systems com-posed of a DM halo and of a stellar population.  Rotation in these galaxies is negligible, and the stellar component is supported against gravity by its random motion.Therefore the observation that can be used to test DM models is not rotation curves but, rather, the line-of-sight velocity dispersions.Walkeret al[30, 34, 35] reported updated empirical ve-locity dispersion profiles for the eight “classical” dSphsof the Milky Way:  Carina, Draco, Fornax, Leo I, LeoII,Sculptor, Sextans, and Ursa Minor; see Figure 1 for de-tails.   Following  standard  parametric  analysis  [30,  31](see Ref. [32] for a different approach), we consider thatthe stellar component in each individual galaxy is in dy-namical  equilibrium  and  that  it  traces  the  underlyingDM  distribution.   Assuming,  further,  spherical  symme-try, Jeans’s equation relates the mass profile of the DM halo,
\begin{equation}
 M(r)={\frac{M_{max}}{\pi}}{\left[\sin{\left(\frac{\pi r}{r_{max}}\right)}−{\left({\frac{\pi r}{r_{max}}}\right)}\cos{\left({\frac{\pi r}{r_{max}}}\right)}\right]},
\end{equation}
where
\begin{equation}
M_{max}=M(r_{max}) = (4/\pi)\rho cr^3_{max},
\end{equation}
\end{multicols}
\begin{figure}[h]
\centering
\includegraphics[width=1\textwidth]{1.jpg}
\caption{\label{fig:1}Empirical, projected velocity dispersion profiles for the classical eight dSph satellites of the Milky Way as reported inRefs. [30, 34, 35].  Solid lines denote the best fits for the halo model in Eq. (2) whenrmax= 1 kpc (red),rmax= 2 kpc (black),andrmax= 6 kpc (blue).}
\end{figure}
\begin{multicols}{2}
to the first moment of the stellar distribution function, 
\begin{equation}
 {\frac{1}{v}}{\frac{d}{dr}}{(\nu {\langle v^2_r\rangle})}+ {2\beta{\langle v^2_r \rangle}r}= −{\frac{GM}{r^2}}.
\end{equation}
Above,$\nu(r)$, $\langle v^2_r(r)\rangle$,  and $\beta(r) = 1−\langle v^2_\theta \rangle/\langle {v_r^2}\rangle$are the three-dimensional density, radial velocity dispersion, and orbital anisotropy, respectively, of the stellar component.The  parameter$\beta$ quantifies  the  degree  of  radial  stellar anisotropy:  if all orbits are circular$\langle v^2_r \rangle = 0$, and then $\beta =\infty$; if the orbits are isotropic $\langle v^2_r \rangle =\langle v^2_\theta \rangle$, and $\beta= 0$;finally,  if  all  orbits  are  perfectly  radial,$\langle v^2_\theta\rangle=  0$,  then$\beta= 1$.  There is no preference a priori for either radially,$\beta >0$, or tangentially,$\beta <0$, biased systems; however,configurations with $\beta \sim1$ are disfavored due to the very particular initial conditions they seem to require.In   the   simplest   scenario   with   constant   orbital anisotropy,$\beta(r)  =$  const,  the  (observed)  projection  of the  velocity  dispersion  along  the  line-of-sight,$\sigma^2_{los}(R)$,relates the mass profile,$M(r)$, to the (observed) stellar density,$I(R)$, through [36]
\begin{equation}
{\sigma^2_{los}}={\frac{2G}{I(R)}}{\int_{R}^{\infty}dr'\nu(r')M(r')(r')^{2\beta−2}F(\beta,R,r')}.
\end{equation}
Here
\begin{equation}
F(\beta,R,r') \equiv {\int_{R}^{r'}dr\left(1−\beta{\frac{R^2}{r^2}}\right){\frac{r^{-2\beta +1}}{\sqrt{r^2-R^2}}}},    
\end{equation}
and R is  the  projected  radius.   We  adopt  a  Plummer profile for the stellar density,
\begin{equation}
I(R) ={\frac{L}{\pi r^2_{half}}}{\frac{1}{[1+(R/r)^2]^2}},   
\end{equation}
where L is  the  total  luminosity  of  the  object  and $r_{half}$ (the only single shape parameter) the half-light radius.The values of these two quantities for each of the eight classical dSphs are listed in Table I of Ref. [30].  Under the  assumption  of  spherical  symmetry  the  corresponding three-dimensional stellar density associated with the Plummer profile takes the form 
\begin{equation}
\nu(r) ={\frac{3L}{4\pi r^3_{half}}}{\frac{1}{[1+(r/r_{half})^2]^{5/2}}}.    
\end{equation}
We have corroborated that our findings in this paper are not very sensitive to the profile of the stellar component,and similar results are also obtained using a Sersic [37]or a King profile [38].
\end{multicols}
\begin{figure}[h]
\centering
\includegraphics[width=1\textwidth]{2.jpg}
\caption{\label{fig:1}Two-dimensional posterior distributions of Fornax, Sculptor, and Draco using the BEC halo model in Eq. (2).  The histograms  correspond  to  the  marginalized  posterior  distributions  of  each  parameter.   The  dashed  lines  and  red  contours represent the $1\sigma$ confidence interval.  Solid lines indicates the maximum likelihood point.}
\end{figure}
\begin{multicols}{2}
\subsection{Maximum Likelihood and Monte Carlo analysis}
In order to fit the observations we have three free parameters per galaxy: two associated with the halo model,the  scale  radius $r_{max}$ and  the  total  mass $M_{max}$,  and one  associated  with  the  stellar  component,  the  orbital anisotropy $\beta$.  Since the scale radius is a constant in the theory  one  could  perform  a  combined  analysis  for  the eight galaxies keeping this quantity fixed.  For the purpose of this paper,  however,  this procedure is not war-ranted;  instead  we  estimate $r_{max}$ for  each  galaxy,  and we  then  compare  the  values  obtained  for  the  different galaxies.  We will also contrast our results against previous constraints arising from the study of the rotational curves  of  LSB  and  dwarf  galaxies.   As  we  show  below,this  analysis  is  sufficient  to  uncover  strong  tension  be-tween model and observations at different scales.In   order   to   proceed   we   perform   a   Maximum Likelihood−Markov chain Monte Carlo analysis (we use the EMCEE code, described in Ref. [39]) to explore the parameter space and estimate the values of $r_{max}$, $M_{max}$ and$\beta$ for each individual galaxy, together with their corresponding uncertainties.  For each galaxy we define the likelihood function
\begin{equation}
\pounds={\prod_{i=1}^{N}{\frac{exp\left[-{\frac{1}{2}}{\frac{(\sigma^{obs}_{los}(R_i)-\sigma_{los(R_i))^2}}{Var[\sigma^{obs}_{los}(R_i)]}} \right]}{\sqrt{2\pi Var[\sigma^{obs}{los}(R_i)]}}}}  
\end{equation}
Here $\sigma^{obs}_{los}(Ri)$ is the observed line-of-sight velocity dispersion at projected radius $R_i$, $\sigma_{los}(R_i)$ is given in Eq. (10),$Var[\sigma^{obs}_{los}(R_i)]$ is the square of the error associated with the observed value of the velocity dispersion at $R_i$, and $i$ is a label for the data bins that runs from 1 to the total number of bins N.  To account for the uncertainties on $r_{half}$ we marginalize over this parameter by sampling it, at each step of the Monte Carlo, from a normal distribution  with  a  standard  deviation  equal  to  its  actual uncertainty.
For  the  three  free  parameters  we  adopt  uniform  log-priors in the following ranges:
\begin{eqnarray}
−2.5<ln(r_{max}[kpc])<2.5\\
−7<ln(M_{max}[10^9M])<7\\
−3<−ln (1−\beta)<3.   
\end{eqnarray}
For each galaxy we run 50 chains simultaneously, starting at random values within the prior range, and allow each chain to run for 1,000 steps, from which we eliminate the first 100 steps that correspond to a “burn-in” period.
\section{RESULTS}
Our  results  are  shown  in  Figure  2  where,  for  three of the galaxies with more data points, Fornax, Sculptor and  Draco,  we  plot  the  one-  and  two-dimensional  posterior  distributions  of  the  parameters $r_{max}$, $M_{max}$,  and $\beta$.  As we can note the posterior distributions are almost symmetric with respect to the maximum likelihood point(solid lines).  The dashed lines and red ellipses indicate the $1\sigma~(68.2\% C.L.)$ confidence interval of the different parameters.  Some degeneracy between the scale radius and the total mass, and the anisotropy, is evident; how-ever, in all cases the chains converge to a small region of the parameter space.The values of $r_{max}$, $M_{max}$ and $\beta$ for all the galaxies in the sample, together with their corresponding uncertain-ties, are listed in Table I. We have corroborated that similar results are also obtained when using a Sersic (King)stellar distribution.  In particular,  for $r_{max}$ we obtain a difference  of $\sim0.5 kpc  (\sim0.2kpc)$  in  the  central  value, but  the  error  remains  of  the  same  magnitude  with  respect to that in the Plummer case.We conclude that the preferred value of the scale radius inferred from the dynamics of the eight dSphs lies around
\end{multicols}
\begin{table}[h]
    \centering
    \begin{tabular}{l c c c}
    \hline
    Object & $r_{max}[kpc]$ & $M_{max}[10^8M]$ & $-ln(1-\beta)$\\
    \hline
    Fornax & $1.4^{+0.1}_{−0.1}$ & $1.1^{+0.9}_{−0.9}$ & $0.2^{+0.1}_{−0.1}$\\
    Sculptor & $1.0^{+0.1}_{−0.1}$ & $1.1^{+0.3}_{−0.2}$ & $0.3^{+0.2}_{−0.2}$\\
    Carina & $1.1^{+0.3}_{−0.3}$ & $0.8^{+0.6}_{−0.3}$ & $0.6^{+0.3}_{−0.3}$\\
    Draco & $1.7^{+0.4}_{−0.3}$ & $5.9^{+4.1}_{−2.7}$ & $1.8^{+0.7}_{−0.8}$\\
    Leo I & $1.0^{+0.4}_{−0.2}$ & $1.7^{+1.4}_{−0.7}$ & $0.9^{+0.7}_{−0.5}$\\
    Leo II & $0.6^{+0.3}_{−0.2}$ & $0.5^{+0.7}_{−0.3}$ & $1.6^{+0.9}_{−0.9}$\\
    Sextans & $0.7^{+0.4}_{−0.3}$ & $0.2^{+0.2}_{−0.1}$ & $0.4^{+0.4}_{−0.7}$\\
    Ursa Minor & $0.9^{+0.4}_{−0.3}$ & $0.9^{+0.9}_{−0.4}$ & $0.1^{+0.3}_{−0.3}$\\
    \hline
    \end{tabular}
    \caption{stimate of the parameters $r_{max}$, $M_{max}$, and $\beta$ for the classical dSphs in the Milky Way.}
    \label{TABLE:1}
\end{table}
\begin{multicols}{2}
$r_{max}\sim1 kpc$, i.e. $m/\lambda^{1/4}sim7eV/c^2$; this value is indeed contained within the $3\sigma(99.7\% C.L.)$ confidence interval of each galaxy. Moreover, we can exclude at more than $5\sigma(99.9\% C.L.)$ values of $r_{max}\gtrsim 5kpc$.  As we will discuss next  in  Section  IV,  this  implies  a  strong  conflict  with previous constraints on this parameter of the theory.At  this  point  we  would  like  to  stress  that,  besidesthe  statistical  evidence  for  small  values  of  the  parameter $r_{max}$, there are also physical arguments that support this conclusion, which we can draw by looking at the behavior of the best fit parameters (minimum chi-square)of the anisotropy, $\beta$,  and total mass,$M_{max}$,  for a fixed value $r_{max}$ of the size of the condensate:(i) Density profiles with scale radii larger than 2 kpc imply  values  of  the  anisotropy  parameter $\beta\gtrsim0.5$;  see Figure 3.  For a scalar field DM model there is no known connection between the anisotropy in the stellar distribution and the halo, so that dSphs could in principle be described as equilibrium systems even with such large val-ues of the orbital anisotropy.  (It is unclear to us whether large  values  of  the  stellar  anisotropy  would  necessarily develop a radial instability for these halo models.)  How-ever,  although  these  configurations  cannot  be  excluded a priori,  they  imply  an  unnatural  preference  for  radial orbits.(ii) As the value of the scale radius increases, the total mass required to fit the data grows drastically, reaching values as large as $M_{max}\gtrsim10^{10}M$ in some cases when $r_{max}\gtrsim6kpc$; see  Figure  4.   This  value  is  an  order  of magnitude larger than what inferred by previous analysis [30, 31, 34, 40].  An upper limit to the mass of these objects stems from the requirement that the dynamical friction decay time not be larger than the age of the uni-verse [36, 41], although there are no model independent limits on the total mass of these galaxies.Finally, it is also interesting to note that observations suggest  a  decline  in  the  velocity  dispersion  profiles  at large projected radii [30, 42], whereas the predicted pro-files for large values of the scale radius grow at large radii.Even though for some galaxies the fit is not drastically worsen for large values of $r_{max}$, if we inspect the overall radial dependence we can see that large scale radii fail in describing the outer regions for all galaxies, see the blue lines in Figure 1. 
\end{multicols}
\begin{figure}[h]
\centering
\includegraphics[width=1\textwidth]{3.jpg}
\caption{\label{fig:1} Preferred  orbital  anisotropy  for  the  best  fits  as  a function of the scale radius.  The lines at $\beta= 0.5$ and $\beta=−1$ correspond to ${\langle v^2_r\rangle}= 2{\langle {v^2_\theta}\rangle}$ and ${\langle {v^2_\theta} \rangle}= 2{\langle v^2_r\rangle}$, respectively.}
\end{figure}
\begin{multicols}{2}
 From the above considerations the preference of a scale radius  in  the  range $r_{max}\sim0.5−2kpc$ (green  band  in Figures 3 and 4) is solid.  A common value of $r_{max}$ larger than 5kpc is clearly disfavored by the dynamics of dSphs.
 \end{multicols}
\begin{figure}[h]
\centering
\includegraphics[width=1\textwidth]{4.jpg}
\caption{\label{fig:1} Total mass for the best fits as a function of the scale radius.  The line at $M= 3×10^9M$ corresponds to the virial mass of Draco (the most massive object in the sample) obtained from a NFW profile consistent with the observations in the velocity dispersions [34, 40].  The line at $M= 1×10^{10}M$comes from an upper limit to the mass of this same galaxy as required from the dynamical friction decay time to be larger than one Hubble time [36, 41].}
\end{figure}
\begin{multicols}{2}
\section{DISCUSSION AND CONCLUSIONS}
The  viability  of  the  halo  model  in  Eq.  (2)  has  been studied  in  several  papers  mainly  employing  rotational
6 curves  of  galaxies  from  different  surveys,  out  of  which only  the  most  DM-dominated  objects  have  been  selected  [8,  16,  18,  19];  see  also  Ref.  [20,  21]  for  a  different  approach.These  studies  all  point  to  a  scale radius  that  varies  from  galaxy  to  galaxy  and  ranges from  3 kpc  up  to  15 kpc  (light  red  band  in  Figures  3 and 4), with only isolated instances requiring values outside  this  range,  e.g.   M81dw,  where $r_{max}\sim1kpc$  [8],and  UGC5005,  where $r_{max}\sim24.65kpc $ [18].   However,these papers also report mean values in the narrow range $r_{max}\sim5.5−7kpc$ [16, 18] (red band in Figures 3 and 4),suggesting the existence of a self-interacting scalar particle with $m/\lambda^{1/4}\sim2.6−2.9eV/c^2$.  Such findings have led  to  the  conclusion  that  the  halo  model  in  Eq.  (2)can describe accurately the dynamics of DM-dominated galaxies.  The case of Milky Way-like systems,  or giant ellipticals,  remains  to  be  studied  in  detail  mainly  be-cause the dynamical interaction between the condensate and  baryons  is  not  well  understood  there  (see  however Ref.  [22],  where  a  set  of  three  high-surface-brightness spirals have been recently considered, e.g.  ESO215G39,where $r_max\sim50kpc$, and Ref. [23], where values of the scale  radius  in  the  range $r_{max}=5.6−98.2kpc$  are  reported for a sub sample of galaxies in the THINGS survey).  Note that, contrary to other proposals in the literature, the halo model in Eq. (2) is not expected to describe galaxy clusters.The  values  reported  in  previous  studies  are  strongly disfavored by our findings in the present analysis, where we show that the dynamics of the smallest and least luminous galaxies is clearly in conflict, along several lines,with such large scale radii.  One could argue that the pro-file in Eq. (2) is not appropriate to describe the galaxies in Refs. [8, 16, 18, 19, 22, 23] (where in some cases the luminous matter extends up to 10 kpc), and suggest that a more elaborated halo model where the condensate rep-resents  only  the  core  of  the  galaxy  would  be  necessary in  order  to  understand  the  dynamics  of  these  systems.However, it is important to note that a condensate with a scale radius of the order of 1 kpc does not provide the core expected for those galaxies used in previous analysis.Interestingly,  the  new  value  of $r_max\sim1kpc$  is favored by cosmological observations.  A homogeneous and isotropic distribution of matter satisfying the Eq. (1) has two  different  regimes  depending  on  the  actual  value  of the charge density,$q=Q/a^3$; see e.g.  Ref. [15].  Here a is the scale factor and Q the number of particles per unit volume today,$a= 1$.  When the charge density is high, $q\gg m^3^c3/(\lambda h^3)$, the energy density and pressure of the scalar  field  dilute  with  the  cosmological  expansion  like dark radiation,$\rho= 3\lambda^{1/3}Q^{4/3}ch/(4a^4)$ and $p= (1/3)\rho$, where as  at  low  densities,  when $q\gg m^3c^3/(\lambda h^3)$,  like cold DM,$\rho=Qmc^2/a^3$ and $p= 0$.  From the condition that the transition from dark radiation to DM, fixed at $q\sim m^3c^3/(\lambda h^3)$, has to occur before the time of equality,when $\rho \sim 5×10^{13}eVcm^{−3}$,  we  obtain $r_{max}<80kpc$, i.e. $m/\lambda^{1/4}>0.8eV/c^2$.   However,  the  number  of  extra relativistic species at Big-Bang nucleosynthesis places tighter constraints on the parameters of the theory.  Fora scalar field, this quantity, defined as the number of extra relativistic neutrino degrees of freedom at Big-Bang nucleosynthesis, takes the for
\begin{equation}
\Delta N_{eff}= {57.83 ×  {(\Omega_{dm}h^2)}^{4/3}}{(\frac{\lambda^{1/4}}{m[eV/c^2]^{4/3}})},   
\end{equation}
see   e.g. Eq.   (67)   in   Ref.   [16].(Note   thatthere   is   an   extra   factor   of   1/2   in   our   expression for  $\Delta N_{eff}$ with   respect   to   that   in Ref.   [16];   this might   come   from   the   two   helicities   of   the   neutrino.)   Using  the  latest  cosmological  data  provided  by PLANCK+WP+highL  [43], $\Omega_{dm}h^2=  0.1142±0.0035$ at  $1\sigma$ C.L.,  and  PLANCK+WP+highL+(D/H)p[44], $\Delta N_{eff}= 0.23±0.28$, also at $1\sigma$CL, we obtain
\begin{equation}
r_{max}\lesssim3kpc,~~i.e.~~m/\lambda^{1/4}\gtrsim4eV/c^2, 
\end{equation}
Note that this result excludes marginally previous values of $r_{max}\gtrsim5.5kpc$ arising from the study of the rotational curves of LSB and dwarf galaxies [8, 16, 18, 19].The  analysis  in  this  paper  applies  only  for  the  case of  a  self-interacting  scalar  particle  with $\lambda >0$;  how-ever,  similar  results  are  expected  when $\lambda\geq0$.   Up  to our  knowledge,  there  is  no  analytic  expression  for  the mass  density  profile  of  the  halo  model  when  the  self-interaction  term  is  less  than  or  equal  to  zero,  but  e.g.in  the  case  of $\lambda= 0$,  the  characteristic  size  and  mass of the equilibrium configurations are found to be [45] of order $R\sim Q^{−1/2}(m_{Planck}/m)(h/mc)$, and $M\sim Qm$, respectively.  One can fix the number of particles,Q, and mass parameter,m, in order to describe the dynamics of dSphs, implying $R\sim1kpc$ and $M\sim10^8M$, see for in-stance Ref. [20] for the case of Ursa Minor, but then configurations heavier than $10^8M$ would be smaller than 1 kpc,  where as  those  larger  than  1 kpc  would  result  in halos lighter than 108M.In summary, if we dismiss previous constraints, a scenario  where  the  DM  galactic  halos  are  described  by  a single  condensate  is  consistent  with  the  data  from  the smallest  and  most  DM-dominated  nearby  galactic  systems; nonetheless, these single objects alone will not be consistent with the description of bigger galaxies

\end{multicols}
\bibliographystyle{alpha}
\bibliography{sample}

\end{document}